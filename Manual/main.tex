\documentclass{article}
\usepackage[utf8]{inputenc}
\usepackage[spanish]{babel}
\usepackage{listings}
\usepackage{graphicx}
\usepackage{float}

\graphicspath{ {imagenes/} }
\usepackage{cite}

\begin{document}

\begin{titlepage}
    \begin{center}
        \vspace*{1cm}
            
        \Huge
        \textbf{Parcial 2}
            
        \vspace{0.5cm}
        \LARGE
        
            
        \vspace{1.5cm}
            
        \textbf{Juan Esteban Garcia Durango\\
            Jessica Valentina Gaviria Samboni }
        \vfill
            
        \vspace{0.8cm}
            
        \Large
        Despartamento de Ingeniería Electrónica y Telecomunicaciones\\
        Universidad de Antioquia\\
        Medellín\\
        Septiembre de 2021
            
    \end{center}
\end{titlepage}

\tableofcontents
\newpage
\section{Manual de uso del programa}\label{intro}
\subsection{Guia para el usuario}


   
\section{Paso a paso} \label{Descripción}
1. Al ingresar a qt lo primero que debe hacer es escribir el nombre de la imagen que desea procesas, acompañada de la extension .jpg , ademas de esto debe tener en cuenta que dentro de la carpeta en la cual esta guardado el codigo debe de haber dos carpetas adicionales, una llamada imagenes y otra llamada BD.\\

2. Es importante tener en cuenta que independiente de las dimensiones de la imagen ingresada, siempre se generara la informacion para una matriz de 8*8.\\

3.El txt generado con la informacion RGB, en este caso un arreglo, debe ser copiada tal cual en el tinkercad, en la parte que se indica en el mismo.\\

4.El programa es capaz de procesar imagenes de cualquier medida.




\bibliographystyle{IEEEtran}

\end{document}

